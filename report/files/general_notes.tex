\section{General notes} 
\begin{itemize}
	\item Exercises 1-4 are done with the ROBUST matlab main and unity visualization
		tools. Exercises 5-6 are done with the DexROV matlab main and unity
		visualization tools.
	\item Comment and discuss the simulations, in a concise scientific manner. Further
		comments, other than the questions, can be added, although there is no need to
		write 10 pages for each exercise.
	\item Aid the discussion with screenshots of the simulated environment (compress
		them to maintain a small overall file size), and graphs of the relevant
		variables (i.e. activation functions of inequality tasks, task variables, and so
		on). Graphs should always report the units of measure in both axes, and legends
		whenever relevant.
	\item Report the thresholds whenever relevant.  Report the mathematical formula
		employed to derive the task jacobians and the control laws when asked, including
		where they are projected.
	\item If needed, part of the code can be inserted as a discussion reference.
\end{itemize}

Use the following template when you need to discuss the hierarchy of tasks of a
given action or set of actions:
\begin{table}[htb]
	\caption{
		Example of actions/hierarchy table: a number in a given cell
		represents the priority of the control task (row) in the
		hierarchy of the control action (column). The type column
		indicates whether the objective is an equality (E) or inequality
		(I) one.
	}
	\label{table:actions_table}
	\begin{center}
		\footnotesize
		\begin{tabular}{ccccc}
			\toprule
			Task & Type & $\mathcal{A}_{1}$ & $\mathcal{A}_{2}$ &
			$\mathcal{A}_{3}$\\
			\midrule
			Task A & I & 1 & & 1   \\
			\hdashline
			Task B & I & 2 & 1 &   \\
			\hdashline
			Task C & E &   & 2 & 2 \\
			\bottomrule
		\end{tabular}
	\end{center}
\end{table}%
