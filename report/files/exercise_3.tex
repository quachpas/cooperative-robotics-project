	\section{Exercise 3: Improve the “Landing” Action}

	\subsection{Adding an alignment to target control objective}
	\question
	If we use the landing action, there is no guarantee that we land in from of the
	nodule/rock. We need to add additional constraints to make the vehicle face the
	nodule. The position of the rock is contained in the variable
	\vect{rock\_center}.

	Initialize the vehicle at the position:

	\begin{displaymath}
		\vec{p}
		=
		\begin{bmatrix}
			8.5 & 38.5 & -36 & 0 & -0.06 & 0.5
		\end{bmatrix}^\top
	\end{displaymath}

	Use a "safe waypoint navigation action" to reach the following position:

	\begin{displaymath}
		\vec{vehicleGoalPosition}
		=
		\begin{bmatrix}
			10.5 & 37.5 & -38 & 0 & -0.06 & 0.5
		\end{bmatrix}^\top
	\end{displaymath}

	Then land, aligning to the nodule.

	Goal: Add an alignment task between the longitudinal axis of the vehicle ($x$
	axis) and the nodule target. In particular, the $x$ axis of the vehicle should
	align to the projection, on the inertial horizontal plane, of the unit vector
	joining the vehicle frame \cframe v to the nodule frame \cframe t.
	\begin{parts}
		\part{Report the hierarchy of tasks used and their priorities in
		each action. Comment the behaviour.}

		\begin{solutionorbox}
			The hierarchy of tasks used is described in
			Table~\ref{table:tkip_swn_improved_landing_robust}.
			
			The Alignement to Target task is also a prerequisite for the
			landing task, along with the horizontal attitude.
			However, its priority is lower than the Horizontal
			attitude task as the vehicle's attitude is a matter of
			safety.

			During the Landing phase, the vehicle first aligns
			itself in terms of position, then in orientation. When
			orienting itself, the descent begins at the same time.
			This behaviour matches the task priorities, as linear
			velocities are first generated for the Alignment to
			Target, then for the Altitude.
		\end{solutionorbox}
		\begin{table}[htb]
			\caption{Hierarchy of tasks for Safe Waypoint Navigation
			\& Improved Landing}
			\label{table:tkip_swn_improved_landing_robust}
			\begin{center}
				\footnotesize
				\begin{tabular}{ccccc}
					\toprule
					Task & Type & Safe Waypoint Navigation &
					Improved Landing
					\\
					\midrule
					Minimum altitude & I & 1 & \\
					\hdashline
					Horizontal attitude & I & 2 & 1\\
					\hdashline
					Alignement to Target & I & & 2\\
					\hdashline
					Vehicle position & I & 3 & \\
					\hdashline
					Altitude & E &  & 3\\
					\bottomrule
				\end{tabular}
			\end{center}
		\end{table}%


		\part{What is the Jacobian relationship for the Alignment to
		Target control task? How was the task reference computed?}

		\begin{solutionorbox}

			Let $\vec{O_v}$ and $\vec{O_t}$ be the respective origins of the
			vehicle frame \cframe v and target frame \cframe t.
			Let $\vec{i_v}$ be the unit vector of the $x$-axis of
			the vehicle, and $\theta$ the positive angle between
			\vect[w]{i_v} and the projected distance vector
			$\vec[w]{d}=\vec[w]{O_t} - \vec[w]{O_v}$ on the inertial
			horizontal plane. Figure \ref{fig:ex3_ta} illustrates
			this task.

			In order to align the vehicle to the target, we define
			the misalignement vector $\vec[w]{\rho}=\theta
			\vec[w]{n_{at}}$, where $\vec[w]{n_{at}}$ is a vector
			along the axis of rotation such that
			$\vec[w]{n_{\at}}=\frac{1}{\sin\theta}\vec[w]{i_v}
			\wedge\left(\frac{\vec[w]{d}}{\lVert\vec[w] d \rVert
			}\right)$.

			In order to align the vehicle, we define our task with
			the objective of minimising $\vec\rho$ using its
			derivative w.r.t. $\theta$. The partial derivative is

			\[
				\frac{\partial \vec[w]\rho}{\partial\theta} =
				\vec[w]{n_{\at}}  \dot{\theta}
				\mathrm{.}
			\]

			Let us express the above expression in terms of $\ydot$.
			First, for $\dot{\theta}$
			\[
				\begin{aligned}
					\dot{\theta} &= \vec[w]{n_{at}}
					\vec\omega_{t/x}\\
						     &= \vec[w]{n_{at}}  \left(
							     \vec\omega_{t/w}
							     -
						     \vec\omega_{x/w}\right)
				\end{aligned}
				\mathrm{.}
			\]

			$\vec[w]{i_v}$ is a unit vector of the $x$-axis of the vehicle
			frame \cframe v, thus we can write $\vec\omega_{v/w}=\vec\omega_{x/w}$.

			Next, for $\vec\omega_{t/w}$
			\[
				\begin{aligned}
					\vec\omega_{t/w} &=
					\frac1{\lVert\vec[w] d\rVert^2}\left(\vec[w] d
					\wedge \vec v_{d/w}\right)\\
							 &=
							 -\frac1{\lVert\vec[w] d\rVert^2}\left(\vec[w] d
								 \wedge \vec
							 v_{v/w}\right)
				\end{aligned}
				\mathrm{,}
			\]
			where $\vec v_{d/w}=\vec v_{t/w} - \vec v_{v/w}$ by
			definition. We assume the target is immobile, thus $\vec
			v_{t/w}=0$.

			Finally, the Jacobian relationship is
			\[
				\dot{\theta} = \vec[w]{n_{at}}\T
				\begin{bmatrix}
					\Znm37\quad
					-\frac1{\lVert\vec[w]
					d\rVert^2}\skewmat[w]
					d\quad
					-\Inm33
				\end{bmatrix}
				\ydot
				= \J[w]_{\at}\ydot
				\mathrm{.}
			\]
		\end{solutionorbox}

		\begin{figure}[h]
			\begin{center}
				\begin{tikzpicture}
					\coordinate (Ov) at (0,0);
					\coordinate (Ot) at (2,2);
					\coordinate (x) at (0,1);

					\filldraw (Ov) circle (2pt) node[left,
						blue] {$O_v$};
					\filldraw[red] (Ot) circle (2pt)
						node[above] {$O_t$};


					\draw[->] (Ov)--(x) node[above]{$x$};
					\draw[->] (Ov)--(Ot) node[midway, below]{$d$};
					\pic [draw, ->, "$\theta$", angle eccentricity=1.5] {angle=Ot--Ov--x};
			\end{tikzpicture}
		\end{center}

		\caption{Target alignement on the inertial horizontal
		plane}
		\label{fig:ex3_ta}
	\end{figure}

	\begin{solutionorbox}
		We compute the misalignement vector \vect{\rho} using the
		Reduced Versor Lemma. The task reference is computed w.r.t.
		$\vec[w]{\rho_{at}}$, as when the vectors are aligned,
		$\lVert\vec[w]{\rho_{at}}\rVert$ tends toward $0$. We have
		\[
			\xdotbar[w]\at =
			\lambda\left(0 - \lVert\vec[w]{\rho_{at}}\rVert\right) \quad
			\lambda\in\mathbb{R}^+
			\mathrm{.}
		\]
	\end{solutionorbox}

	\part{Try changing the gain of the alignment task. Try at least
		three different values, where one is very small. What is the observed
		behaviour? Could you devise a solution that is gain-independent guaranteeing
	that the landing is accomplished aligned to the target?}

	\begin{solutionorbox}
		The smaller the gain, the slower the vehicle orients itself to
		the target. However, regardless of the gain, the vehicles
		changes its position in order to align to the target, albeit
		slower the smaller the gain. For a very small gain, the vehicle
		does not seem to orient itself at all.

		In order to have a gain-independent behaviour, we can add a bias term
		to the task reference.
		\[
			\xdotbar[w]\at =
			\lambda\left(0 - \lVert\vec[w]{\rho_{at}}\rVert\right) -
			\mu\quad
			\lambda,\mu\in\mathbb{R}^+
			\mathrm{.}
		\]
	\end{solutionorbox}

	\part{After the landing is accomplished, what happens if you try
		to move the end-effector? Is the distance to the nodule sufficient to reach it
		with the end-effector? Comment the observed behaviour. If, after landing, the
		nodule is not in the manipulator's workspace, what was missing in the previous
	actions? How would you fix it?}

	\begin{solutionorbox}
		After landing, the end-effector moves towards the target,
		however the distance to the nodule is insufficient to reach it
		with the end-effector. The end-effector stays elongated, but
		does not touch the nodule.

		If after landing, the target is not within the manipulator's
		workspace, the TKIP is missing a task involving the distance to
		the target.

		To fix the Landing action, we would add a new inequality task
		called Distance to Target, which makes use of the distance
		vector \vect[w]{d} to reach the objective distance value. The
		new hierarchy of tasks is described in
		Table~\ref{table:tkip_swn_improved_landing_v2_robust}, where the
		Distance to Target task priority is below both Alignement to
		Target and Horizontal attitude.
	\end{solutionorbox}
	\begin{table}[htb]
		\caption{Hierarchy of tasks for Safe Waypoint Navigation \&
		Improved Landing v2 - ROBUST}
		\label{table:tkip_swn_improved_landing_v2_robust}
		\begin{center}
			\footnotesize
			\begin{tabular}{ccccc}
				\toprule
				Task & Type & Safe Waypoint Navigation &
				Improved Landing v2
				\\
				\midrule
				Minimum altitude & I & 1 & \\
				\hdashline
				Horizontal attitude & I & 2 & 1\\
				\hdashline
				Alignement to Target & I & & 2\\
				\hdashline
				Distance to Target & I & & 3\\
				\hdashline
				Vehicle position & I & 3 & \\
				\hdashline
				Altitude & E &  & 4\\
				\bottomrule
			\end{tabular}
		\end{center}
	\end{table}%
\end{parts}
