	\section{Exercise 2: Implement a Basic “Landing” Action.}
	\subsection{Adding an
	altitude control objective}
	\question
	Initialize the vehicle at the position:

	\begin{displaymath}
		\vec{p}
		=
		\begin{bmatrix}
			10.5 & 37.5 & -38 & 0 & -0.06 & 0.5
		\end{bmatrix}\T
	\end{displaymath}

	Goal: add a control task to regulate the altitude to zero.
	\begin{parts}
		\part{Report the hierarchy of task used and their priorities to
			implement the Landing Action. Comment how you choose the priority level for the
		altitude control task.}

		\begin{solutionorbox}
			The hierarchy of tasks for the Landing is described in
			Table~\ref{table:tkip_landing_robust}, where the horizontal
			attitude task has the highest priority.

			Given the vehicle, the landing is safer to perform if
			the vehicle is horizontal to the seafloor, when it is
			assumed to be relatively flat. The horizontal attitude
			is a prerequisite for the landing action.
		\end{solutionorbox}
		\begin{table}[htb]
			\caption{Hierarchy of tasks for Landing - ROBUST}
			\label{table:tkip_landing_robust}
			\begin{center}
				\footnotesize
				\begin{tabular}{ccccc}
					\toprule
					Task & Type & Landing
					\\
					\midrule
					Horizontal attitude & I & 1 \\
					\hdashline
					Altitude & E & 2 \\
					\bottomrule
				\end{tabular}
			\end{center}
		\end{table}%

		\part{What is the Jacobian relationship for the Altitude control
		task? How was the task reference computed?}

		\begin{solutionorbox}
			The Jacobian relationship for the altitude control task is
			identical to the minimum altitude task.
			\[
				\J[v]_{\alt} = \begin{bmatrix}
					\Znm17\quad
					\vec[v]n\T\quad
					\Znm13
				\end{bmatrix}\mathrm{.}
			\]

			The task reference is equally similar to the minimum
			altitude task, except the objective altitude is in this
			instance $0$.
			\[
				\xdotbar[v]{\alt} = \lambda(0 - \vec{a})
				\quad \lambda\in\mathbb{R}^+
				\mathrm{.}
			\]

		\end{solutionorbox}

		\part{How does this task differs from a minimum altitude control
		task?}

		\begin{solutionorbox}
			This task differs from the minimum altitude control task
			in two points:
			\begin{itemize}
				\item The objective altitude is $0$;
				\item The altitude task has a lower priority
					than the horizontal attitude task.
			\end{itemize}

		\end{solutionorbox}
	\end{parts}

	\subsection{Adding mission phases and change of action}
	\question

	Initialize the vehicle at the position:

	\begin{displaymath}
		\vec{p}
		=
		\begin{bmatrix}
			8.5 & 38.5 & -36 & 0 & -0.06 & 0.5
		\end{bmatrix}\T
	\end{displaymath}

	Use a "safe waypoint navigation action" to reach the following
	position:

	\begin{displaymath}
		\vec{vehicleGoalPosition}
		=
		\begin{bmatrix}
			10.5 & 37.5 & -38 & 0 & -0.06 & 0.5
		\end{bmatrix}\T
	\end{displaymath}

	When the position has been reached, land on the seafloor using the basic
	"landing" action.
	\begin{parts}
		\part{Report the unified hierarchy of tasks used and their
		priorities.}

		\begin{solutionorbox}
			The unified hierarchy of tasks is illustrated in
			Table~\ref{table:tkip_swn_landing_robust}. The Table is read
			left to right, where the Safe Waypoint Navigation is the
			first action, and the Landing the second action.
		\end{solutionorbox}

		\begin{table}[htb]
			\caption{Hierarchy of tasks for Safe Waypoint Navigation
			\& Landing - ROBUST}
			\label{table:tkip_swn_landing_robust}
			\begin{center}
				\footnotesize
				\begin{tabular}{ccccc}
					\toprule
					Task & Type & Safe Waypoint Navigation & Landing
					\\
					\midrule
					Minimum altitude & I & 1 & \\
					\hdashline
					Horizontal attitude & I & 2 & 1\\
					\hdashline
					Vehicle position & I & 3 & \\
					\hdashline
					Altitude & E &  & 2\\
					\bottomrule
				\end{tabular}
			\end{center}
		\end{table}%
		\part{How did you implement the transition from one action to the
		other?}

		\begin{solutionorbox}
			In order to transition from one action to another, a
			trigger condition must be met, which can simply be
			considered as the completion of the current action's
			objective. In practice, we compute an error metric
			of the current parameters against the objective
			parameters and check during the simulation main loop
			whether the error is insignificant enough to consider
			the current action as completed.

			In order to smoothly activate (or deactivate) tasks
			between different actions, we sequence all tasks using a
			new activation function $\app$, such that the modified
			activation function is defined as \[
				a(\vec x, \vec p) = a^i(\vec x) \app
				\mathrm{,}
			\] where $\vec p$ is a vector of variables external to
			the control variables $\vec x$, conveniently
			parameterized by the time elapsed in the current action
			in order to achieve a smooth transition, e.g. a
			continuous increasing/decreasing function between
			$0$ and $1$.

			Once the trigger condition is met for the current
			action, the sequencing activation functions $\app$ are
			enabled for the transition to the next action, which
			starts the smooth transition to the new task priorities.

		\end{solutionorbox}
	\end{parts}
