	\section{Exercise 4: Implementing a Fixed-base Manipulation Action}
	\subsection{Adding non-reactive tasks}
	\question
	To manipulate as a fixed based manipulator, we need to constraint the vehicle to
	not move, otherwise the tool frame position task will make the vehicle move.

	Goal: Add a constraint task that fixes the vehicle velocity to zero. Land on the
	seafloor. Try reaching the rock position with the end-effector, and observe that
	the vehicle does not move.

	\begin{parts}
		\part{Report the hierarchy of tasks used and their priorities in
		each action. At which priority level did you add the constraint task?}

		\begin{solutionorbox}
			The hierarchy of tasks used is described in
			Table~\ref{table:tkip_swn_improved_landing_manipulation_robust}.

			The vehicle null velocity task is added as the
			top-priority task for the Manipulation action, in order
			for subsequent tasks to account for the constraint.
		\end{solutionorbox}

		\begin{table}[htb] 
			\caption{Hierarchy of tasks for Safe Waypoint Navigation
			\& Improved Landing \& Manipulation - ROBUST}
			\label{table:tkip_swn_improved_landing_manipulation_robust}
			\begin{center}
				\footnotesize
				\begin{tabular}{ccccc}
					\toprule Task & Type &
					Safe Waypoint Navigation & Improved
					Landing & Manipulation\\
					\midrule Minimum altitude & I & 1 & &\\
					\hdashline Horizontal attitude & I & 2 & 1& \\
					\hdashline Alignement to Target & I & & 2&\\
					\hdashline Vehicle null velocity & E & & & 1\\
					\hdashline Vehicle position & I & 3 & &\\
					\hdashline Altitude & E &  & 3& \\
					\hdashline End-effector position & I & & & 2\\
					\bottomrule
				\end{tabular}%
			\end{center}%
		\end{table}%

		\part{What is the jacobian relationship for the vehicle null
		velocity task? How was the task reference computed?}

		\begin{solutionorbox}
			We consider simply the Jacobian $\J[v]_{\vnv}$ as
			\[
				\J[v]_{\vnv} = \begin{bmatrix}
					\Znm67\quad
					\Inm66
				\end{bmatrix}
				\mathrm{,}
			\]
			which corresponds to the end-effector joints (zeros
			block) and vehicles velocities (ones block).

			The task reference corresponds to the objective vehicle
			velocities, that is
			\[
				\xdotbar[v]\vnv = \begin{bmatrix}
					\Znm61
				\end{bmatrix}
				\mathrm{.}
			\]
		\end{solutionorbox}

	\end{parts}

	\subsection{Adding a joint limit task}
	\question
	Let us now constrain the arm with the actual joint limits. The vector variables
	\vect{uvms.jlmin} and \vect{uvms.jlmax} contain the maximum and minimum
	values respectively.

	Goal: Add a joint limits avoidance task. Land on the seafloor. Try reaching the
	rock position with the end-effector, and observe that the vehicle does not move
	and that all the joints are within their limits.
	\begin{parts}
		\part{Report the hierarchy of tasks used and their priorities in
		each action. At which priority level did you add the joint limits task?}

		\begin{solutionorbox}
			The hierarchy of tasks used is described in
			Table~\ref{table:tkip_swn_improved_landing_manipulation_v2_robust}
			The Joint limits safety task is added below
			the vehicle null velocity constraint task, and before
			the end-effector position action-defining task.
		\end{solutionorbox}
		\begin{table}[htb] 
			\caption{Hierarchy of tasks for Safe Waypoint Navigation
				\& Improved Landing \& Manipulation v2 - ROBUST}
			\label{table:tkip_swn_improved_landing_manipulation_v2_robust}
			\begin{center}
				\footnotesize
				\begin{tabular}{ccccc}
					\toprule Task & Type &
					Safe Waypoint Navigation & Improved
					Landing & Manipulation v2\\
					\midrule Vehicle null velocity & E & & & 1\\
					\hdashline Minimum altitude & I & 1 & &\\
					\hdashline Horizontal attitude & I & 2 & 1& \\
					\hdashline Joint limits & I & & & 2\\
					\hdashline Alignement to Target & I & & 2&\\
					\hdashline Vehicle position & I & 3 & &\\
					\hdashline Altitude & E &  & 3& \\
					\hdashline End-effector position & E & & & 3\\
					\bottomrule
				\end{tabular}%
			\end{center}%
		\end{table}%
		\part{What is the Jacobian relationship for the Joint Limits task?
		How was the task reference computed?}

		\begin{solutionorbox}
			We consider simply the Jacobian $\J_{\jl}$ as
			\[
				\J_{\jl} = \begin{bmatrix}
					\Inm77\quad
					\Znm66
				\end{bmatrix}
				\mathrm{.}
			\]

			We want the desired task reference to be a value away
			from the limits \vect{uvms.jlmin} and \vect{uvms.jlmax}.
			We choose $\xdotbar\jl$ to reach for the mean value,
			a \emph{center} position,
			\[
				\xdotbar\jl =
				\lambda\left(\frac{\vec{uvms.jlmin} +
					\vec{uvms.jlmax}}2 -
					\vec{q}
				\right)\quad
				\lambda\in\mathbb{R}^+
				\mathrm{,}
			\]
			where \vect{q} are the current joint values.
		\end{solutionorbox}
	\end{parts}
